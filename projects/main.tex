\documentclass[11pt]{article}
\usepackage{blindtext}
\usepackage{geometry}
 \geometry{
 a4paper,
 total={170mm,257mm},
 left=15mm,
 top=20mm,
 }
\usepackage{cmbright}
\usepackage{bm}
\usepackage{graphicx}
\usepackage{amsmath,amssymb}
\usepackage[citecolor = green, linkcolor = black, bookmarks=true, urlcolor=gio,
colorlinks=true, pagebackref=true]{hyperref}
\usepackage[T1]{fontenc}
\usepackage[english]{babel}
\usepackage{enumitem}


\title{Project plans}
\author{}
\date{}

\begin{document}

\maketitle
\tableofcontents

% ---------- INTERFACE
\section{Interface}

% ---------- optimizers refactor
\subsection{Optimizers refactor}
The optimization module will be extended, introducing more options like state-of-art
heuristic algorithm and gradient based optimizers. The current shape of Qibo relies on 
TensorFlow to compute automatic differentiation, in particular in Quantum Machine 
Learning routines. We will upgrade this aspect by using hardware-compatible differentiation 
rules to make the calculations on the quantum devices possible, integrating this 
strategy with the current optimization layout.

\subsubsection*{Deliverables and milestones}
\begin{itemize}[noitemsep]
\item add new optimizers (Basin-Hopping, Simulated Annealing, hardware-compatible
gradient descent, etc) [by Nov. 2023];
\item write custom derivation rules for quantum circuits, integrating them with 
the TensorFlow backend. These will be hardware-compatible by deploying the 
Parameter Shift Rules [by Dec. 2023];
\item make Qibo's interface compatible with other frameworks (JaX, Pytorch) [by Apr. 2024]
\item lighten the access interface to optimisers [by Feb. 2023];
\end{itemize}

% ---------- Qiboml refactor
\subsection{Qiboml}
A new Qibo module will be developed to perform Quantum Machine Learning tasks. 
It will provide the users with a simple interface to write QML prototypes and algorithms
using Qibo and to freely define custom loss functions and optimizers. We will 
focus on a full-stack approach to the QML routines, having as our strength the possibility 
to operate from the highest to the lowest level of the quantum computing thanks 
to the cooperation between Qibo, Qibolab and Qibocal.

\subsubsection*{Deliverables}
\begin{itemize}[noitemsep]
\item define the QML pipeline and thus the interface [by Dec. 2023];
\item specialize the Qibo circuit for QML applications [by Dec. 2023];
\item integrate the Qibo's optimization module with the Qiboml's one [By Feb. 2024]; 
\item write a collection of loss function prototypes [by Mar. 2024];
\item release Qiboml [by Sep. 2024].
\end{itemize}

% ---------- INTERFACE
\section{Simulation upgrades}

% ---------- qibotn
\subsection{Qibotn}
A new tensor network simulation package will be provided, which will built on top
of Qibo. This is crucial to increase the number of simulated qubits, in order to 
treat problems of high interest in Quantum Computing, Quantum Machine Learning
and Quantum Chemistry.  


\subsubsection*{Deliverables}
\begin{itemize}[noitemsep]
\item del 1 [by Jan 1990];
\item del 2 [by Jan 1990];
\item del 3 [by Jan 1990].
\end{itemize}


% ---------- quantum federated learning
\subsection{Quantum Federated Learning}
Federated Learning is a cooperative approach to computational problems. Widely 
used in Machine Learning, it implements a decentralized training of the models, 
making more than one process unit cooperating to get the final common result. 
Thanks to the agnostic structure of Qibo, Qibolab and Qibocal, we aim to implement 
a quantum version of the FL strategy, in which different quantum devices cooperates
to a single QML training. 


\subsubsection*{Deliverables and milestones}
\begin{itemize}[noitemsep]
\item define the QFL interface [by Nov. 2023];
\item define the test problem and prepare the qubits [By Dec. 2023];
\item run the QFL process [By Feb. 2024].
\end{itemize}

% ---------- multi-node
\subsection{Multi-node}
A brief abstract here.


\subsubsection*{Deliverables and milestones}
\begin{itemize}[noitemsep]
\item del 1 [by Jan 1990];
\item del 2 [by Jan 1990];
\item del 3 [by Jan 1990].
\end{itemize}

% ---------- INTERFACE
\section{Validation algorithms}

% ---------- full-quantum PDF fit
\subsection{Determining the proton content with real superconducting qubits}
Quantum computers has been proved to be able to successfully fit the proton content
in the recent years. We aim to deploy the entire fitting process on a real 
superconducting chip. 


\subsubsection*{Deliverables and milestones}
\begin{itemize}[noitemsep]
\item adapt the simulation problem to the hardware execution [by Sep 2024];
\item run the experiment [by Nov 2024].
\end{itemize}

% ---------- train a quantum GAN in the lab
\subsection{Physical pulses to train a Style-Based Quantum GAN}
An hybrid classical-quantum generative model has been presented in 2021. The model 
was trained using the Qibo's tensorflow backend in simulation mode. The new goal 
is to prove the training on the real chip is possible, with the addition of a new
hardware noise source. In fact, we are going to use two superconducting qubits to compute the 
task: one qubit will generate the noise, which will contaminate the pulses executed 
on the second qubit. This second qubit will be in charge of learning the target 
distribution. 


\subsubsection*{Deliverables and milestones}
\begin{itemize}[noitemsep]
\item adapt the simulation problem to the hardware execution [by Nov. 2023];
\item train a dummy model on hardware with the old algorithm [by Dec. 2023];
\item run the experiment [by Dec 2024].
\end{itemize}

% ---------- Boosting VQE with DBF
\subsection{Boosting VQE with Double Braket Flow}
Variational Quantum Eigensolvers (VQEs) are well-known quantum computing algorithms,
which are used to find the ground state of a physical system. We are going to 
boost the VQE execution in Qibo using the Double Braket Flow (DBF) algorithm to 
diagonalize the generator of the system's dynamics. 


\subsubsection*{Deliverables and milestones}
\begin{itemize}[noitemsep]
\item implement a validation VQE prototype [by Dec. 2023];
\item add the DBF procedure to see if the problem scales better with the number 
of qubits [by Apr. 2024].
\end{itemize}


% ---------- Noise Resistend Quantum Neural Networks
\subsection{Noise Resistend Quantum Neural Networks}
Variational Quantum Algorithms (VQAs) are proved to react particularly well to the noise
impact during the training process. With this work we aim to study many techniques 
to improve the robustness of Quantum Neural Network if used in a noisy landscape.
We will focus on adversarial learning and quantum error mitigation strategies 
to define a "good practice" way to initialize a VQA problem.


\end{document}