\documentclass[11pt]{article}
\usepackage{blindtext}
\usepackage{geometry}
 \geometry{
 a4paper,
 total={170mm,257mm},
 left=15mm,
 top=20mm,
 }
\usepackage{cmbright}
\usepackage{bm}
\usepackage{graphicx}
\usepackage{multicol}
\usepackage{fontawesome}

\usepackage[dvipsnames]{xcolor}
\definecolor{light-gray}{gray}{0.95}
\definecolor{elle}{rgb}{0.3, 0.5, 0.9}
\definecolor{gio}{rgb}{0.0, 0.4, 0.65}
\definecolor{def}{rgb}{1, 0.08, 0.58}
\definecolor{purpleheart}{rgb}{0.41, 0.21, 0.61}
\usepackage[citecolor = green, linkcolor = gio, bookmarks=true, urlcolor=gio,
colorlinks=true, pagebackref=true]{hyperref}
\usepackage[T1]{fontenc}
\usepackage[english]{babel}


\newcommand{\cmd}[1]{\colorbox{light-gray}{\textcolor{gio}{\texttt{#1}}}}

\title{Comandi utili e commenti}
\author{Matteo Robbiati}
\date{}

\begin{document}
\maketitle

\begin{multicols}{2}

\tableofcontents

\section{Useful definitions}

Here we collect some useful definitions. \\

\cmd{> the XY problem}: \href{https://xyproblem.info/}{https://xyproblem.info/}

\cmd{> metadata}: data which bring information about other data.

\cmd{> slug}: a slug is the unique identifying part of a web address, typically 
at the end of the URL.

\cmd{> web server}: a web server is software and hardware that uses HTTP 
(Hypertext Transfer Protocol) and other protocols to respond to client requests
 made over the World Wide Web. The main job of a web server is to display website 
 content through storing, processing and delivering webpages to users. Besides HTTP, 
 web servers also support SMTP (Simple Mail Transfer Protocol) and FTP 
 (File Transfer Protocol), used for email, file transfer and storage.

\cmd{> HTTP}: the Hypertext Transfer Protocol (HTTP) is an application layer 
protocol in the Internet protocol suite model for distributed, collaborative, 
hypermedia information systems. HTTP is the foundation of data communication for 
the World Wide Web, where hypertext documents include hyperlinks to other resources 
that the user can easily access, for example by a mouse click or by tapping the 
screen in a web browser.

\section{File extensions}

Here we collect some descriptions of useful file's extensions: \\

\cmd{> *.toml}: TOML aims to be a minimal configuration file format that's easy 
to read due to obvious semantics. TOML is designed to map unambiguously to a hash 
table. TOML should be easy to parse into data structures in a wide variety of 
languages (\href{https://toml.io/en/}{https://toml.io/en/}).

\cmd{> *.yaml}: is a human-readable data-serialization language. It is commonly 
used for configuration files and in applications where data is being stored or 
transmitted.

\cmd{> *json}: aka Javascript Object Notation, is an open standard file format 
and data interchange format that uses human-readable text to store and transmit
 data objects consisting of attribute–value pairs and arrays (or other serializable values). 

\section{Terminal}

\cmd{> ln} create a symbolic link to an object (folder, file, as you want).

\section{\texttt{javascript}}

\cmd{> .babelrc} is the file in which you can customize che babel action. 
\texttt{babel} is a \texttt{javascript} compiler.

\subsection{\texttt{Nextjs}}

\cmd{> getStaticPaths} When you export a function called \texttt{getStaticPaths} 
(Static Site Generation) from a page that uses dynamic routes, Next.js will 
statically pre-render all the paths specified by \texttt{getStaticPaths}.

\subsubsection{\texttt{npm}}
Stand for Node Package Manager, it's the official package manager which is 
installed together with \texttt{node.js}. Help us with installation, managing of 
versions and of dependencies.

\cmd{> npm audit}: search for vulnerabilities in the project you are checking for.

\subsubsection{\texttt{yarn}}

Yarn is a new package manager that replaces the existing workflow for the npm 
client or other package managers while remaining compatible with the npm registry. 

\cmd{> yarn install} is typically required when a new app is initialized or cloned.

\cmd{> yarn dev} launches the development mode: with this command your app is
showed in a local HTTP server and every correct modification to your files is
suddenly rendered on the server.

\section{\texttt{mdx}}

Here we introduce some concept related to \texttt{mdx}, a markdown language which 
is compiled to JavaScript. \\

\cmd{@mdx-js/loader} is needed to compile \texttt{mdx}.

\subsection{rehype}

Rehype is an HTML processor, also powered by a plugin ecosystem. Similar to remark, 
these plugins let you manipulate, sanitize, compile and configure all types of data, 
elements and content.

\subsection{remark}

Remark is a markdown processor powered by a plugins ecosystem. This plugin ecosystem 
lets you parse code, transform HTML elements, change syntax, extract frontmatter, 
and more. Using remark-gfm to enable GitHub flavored markdown (GFM) is a popular option.

\section{Jupyter notebooks}

\cmd{> datas in jupyter notebooks} For every figure, such as a plot, Jupyter 
includes not only the image itself in the notebook, but also a plain text description 
that includes the id (like a memory address) of the object, such as 
\texttt{<matplotlib.axes.\_subplots.AxesSubplot at 0x7fbc113dbe90>}. 
This changes every time you execute a notebook, and therefore will create a 
conflict every time two people execute this cell

\cmd{> \%\% capture <name\_of\_the\_container>} is a magic command which literally 
captures the output provided by the cell execution. All datas are collected into 
an \texttt{Ipython} object called \texttt{<name\_of\_the\_container>} and they 
can be accessed by typing the container, or by calling the specific items like
 elements of the container: \texttt{<name\_of\_the\_container>.item}. 


\section{\texttt{git}}

Some useful commands using \texttt{git}. \\

\cmd{> git fetch} downloads objects and refs from another repository

\cmd{> git pull <branch1> <branch2>} fetches from and integrate with another 
repository or a local branch. It incorporates changes from a remote \texttt{<branch1>} 
status repository into the current \texttt{<branch2>} branch.

\cmd{> git push <remote> <local>} update remote refs along with associated objects. 
Typically you use it for updating a remote branch with modifications you have done 
in \texttt{<local>}.

\cmd{> git add -u} add to the `git` folder all the file you modified.

\cmd{> git reset <commit>} restoring the git files bringing them to the 
\texttt{<commit>} status;

\cmd{> git push $--$force-with-lease} with this command we can push also if we 
want to modify files in the past. The \texttt{git push} command is only \texttt{feed-forward}.

\cmd{> git branch} shows the list of the local branches. Adding \texttt{-r} you 
can display the list of all remote branches. Adding \texttt{-a} you will display 
all branches both remote and local. Adding \cmd{-vva} you will show the verbose 
(detailed) version of all branches description.  

\cmd{> git reset HEAD $\sim$1 <filename>} undos the last commit you have done. 
The default mode is the \texttt{mixed} one, in which the commit is undone but it 
is still saved into the working tree and you must command \texttt{git add <changes>} 
in order to recover the commit. Other options are \texttt{git reset $--$soft HEAD~1}, 
which undoes the commit but saves the "addition" of the files. So you will need 
only to commit again if you want it. Again, you can use \texttt{git reset $--$hard HEAD~1}, 
through which you will lose all uncommited changes and all untracked files 
(with a \texttt{git status} you will not see the modifications in green or in red).

\cmd{> git clean} cleans the working tree by recursively removing files that are 
not under version control, starting from the current directory.

\cmd{> git revert} given one or more existing commits, revert the changes that 
the related patches introduce, and record some new commits that record them. 
This requires your working tree to be clean (no modifications from the HEAD commit).

\subsection{\texttt{git} options}

It can be useful to customize a \texttt{git} command and this can be done by adding 
to the string you write the following options: \\

\cmd{-d} act also on the directories

\cmd{-f} force the action

\subsection{GitHub}

\subsubsection{GitHub actions}

An event is a specific activity in a repository that triggers a workflow run. 
For example, activity can originate from GitHub when someone creates a pull request, 
opens an issue, or pushes a commit to a repository.

\subsubsection{Workflows}

A workflow is a configurable automated process that will run one or more jobs. 
Workflows are defined by a YAML file checked in to your repository and will run 
when triggered by an event in your repository, or they can be triggered manually, 
or at a defined schedule. \\

\cmd{> .github/workflows} is the right place in which the YAML file, which 
describes the workflow, must be placed.

\cmd{> reusable workflow} is a workflow which can be called and used inside 
another workflow.

\section{\texttt{python}}

Here we collect some utils for \texttt{python} programming. \\

\cmd{> python -m http.server} opens a python hosted server in which you can 
visualize the output of, e.g., \texttt{*.html} files.

\cmd{> help(function)} can be used for printing information about a specific 
function.
Try for example with \texttt{help(numpy.asarray)}.

\section{\texttt{pip}}

Here we collect some useful info about the official  package installer for Python. \\

\cmd{pip install} is the official command;

\cmd{pip install -e <package>} installs \texttt{<package>} in editable mode. 
For example, installing \texttt{qibo} using this flag, the usable version of the 
package will be the currently modified one.

\section{\texttt{poetry}}

Here we collect some utils while using \texttt{poetry}. \\

\cmd{poetry update -vvv} must be used when you have changed some deps into e.g. 
\texttt{pyptoject.toml} file. This enables the desired deps updating the 
\texttt{poetry.lock} file.

\section{\LaTeX}

Here we collect some useful comments in the context of the \LaTeX$\,$ use. \\

\cmd{> [fragile]} in the \texttt{beamer} class is used when we treat a text which 
cannot be "interpreted the way text is usually interpreted by \LaTeX".


\newpage

\section*{Useful links}

\cmd{> \href{https://cdnjs.com/libraries/highlight.js}{https://cdnjs.com/libraries/highlight.js}}:
 a collection of useful urls for using pre-coded \texttt{css} files.

\newpage
\end{multicols}

\end{document}


