\documentclass[11pt]{article}
\usepackage{blindtext}
\usepackage{geometry}
 \geometry{
 a4paper,
 total={170mm,257mm},
 left=15mm,
 top=20mm,
 }
\usepackage{cmbright}
\usepackage{bm}
\usepackage{graphicx}
\usepackage{multicol}
\usepackage{fontawesome}

\usepackage[dvipsnames]{xcolor}
\definecolor{light-gray}{gray}{0.95}
\definecolor{elle}{rgb}{0.3, 0.5, 0.9}
\definecolor{gio}{rgb}{0.0, 0.4, 0.65}
\definecolor{def}{rgb}{1, 0.08, 0.58}
\definecolor{purpleheart}{rgb}{0.41, 0.21, 0.61}
\usepackage[citecolor = green, linkcolor = gio, bookmarks=true, urlcolor=gio,
colorlinks=true, pagebackref=true]{hyperref}
\usepackage[T1]{fontenc}
\usepackage[english]{babel}


\newcommand{\cmd}[1]{\colorbox{light-gray}{\textcolor{gio}{\texttt{#1}}}}

\title{PhD notes}
\author{Matteo Robbiati}
\date{}

\begin{document}
\maketitle

\tableofcontents


\section{Computational Complexity Theory}

\subsection{Turing Machine}

A Turing machine is a mathematical model of computation describing an abstract 
machine that manipulates symbols on a strip of tape according to a table of rules. 
Despite the model's simplicity, it is capable of implementing any computer 
algorithm. A Turing machine is a general example of a central processing unit 
(CPU) that controls all data manipulation done by a computer, with the canonical 
machine using sequential memory to store data. More specifically, it is a machine 
(automaton) capable of enumerating some arbitrary subset of valid strings of an 
alphabet; these strings are part of a recursively enumerable set. A Turing 
machine has a tape of infinite length on which it can perform read and write 
operations.

\subsection{Nondeterministic Turing Machine}

In theoretical computer science, a nondeterministic Turing machine (NTM) is a 
theoretical model of computation whose governing rules specify more than one 
possible action when in some given situations. That is, an NTM's next state is 
not completely determined by its action and the current symbol it sees, unlike 
a deterministic Turing machine.

\subsection{NP problems}

In computational complexity theory, NP (nondeterministic polynomial time) is a 
complexity class used to classify decision problems. NP is the set of decision 
problems for which the problem instances, where the answer is "yes", have proofs 
verifiable in polynomial time by a deterministic Turing machine, or alternatively 
the set of problems that can be solved in polynomial time by a nondeterministic 
Turing machine.

\subsection{NP-hard problems}

In computational complexity theory, NP-hardness (non-deterministic 
polynomial-time hardness) is the defining property of a class of problems that 
are informally "at least as hard as the hardest problems in NP". A simple 
example of an NP-hard problem is the subset sum problem.

\section{Quantum computation theory}

We can call quantum channel each transformation which maps a density matrix into
another.

\subsection{Adiabatic Computing}

We encode a problem into an adiabatic evolution context. We start from an initial
hamiltonian $H_0$ and we want to slowly evolve to a final hamiltonian $H_1$. 
If the evolution is slow enough, it is prooved that we continue to stay in the 
ground state of the evolving hamiltonian, falling in the ground state of $H_1$ in 
the end.

\end{document}